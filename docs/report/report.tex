\documentclass{../llncs}
%%%%%%%%%%%%%%%%%%%%%%%%%%%%%%%%%%%%%%%%%%%%%%%%%%%%%%%%%%%
%% package sillabazione italiana e uso lettere accentate
%\usepackage[latin1]{inputenc}
\usepackage[utf8]{inputenc}
%\usepackage[english]{babel}
%\usepackage[italian]{babel}
\usepackage[english,italian]{babel} % documento in italiano con alcune parti in inglese
\usepackage[T1]{fontenc}
%%%%%%%%%%%%%%%%%%%%%%%%%%%%%%%%%%%%%%%%%%%%%%%%%%%%%%%%%%%%%

% Per singole parole o brevi frasi in lingua straniera è disponibile il comando
% \foreignlanguage{lingua}{testo}
% 
% Per porzioni di testo in lingua più consistenti è disponibile l’ambiente
% \begin{otherlanguage*}{lingua}
% ...
% \end{otherlanguage*}

% per gli elenchi
\usepackage{enumitem}

% per la bibliografia
\usepackage[
	autostyle,italian=guillemets
	% ... altre opzioni
]{csquotes}

\usepackage[
	% ... opzioni
	backend=biber
]{biblatex}

%\addbibresource{../bibliografia.bib}

\defbibfilter{papers}{
  type=article or
  type=inproceedings
}

\defbibheading{principali}{\subsection*{Bibliografia essenziale}}
\defbibheading{web}{\subsection*{Fonti e materiali on-line}}
\defbibheading{integrative}{\subsection*{Letture di approfondimento}}

% ambiente per le citazioni, in alternativa a quelli standard (quote e quotation)
\usepackage{quoting}
\quotingsetup{font=small}

% per usare \caption* per didascalie senza intestazione e numero
\usepackage{caption}

% https://en.wikibooks.org/wiki/LaTeX/Source_Code_Listings
%%%%%%%%%%%%%%%%%%%%%%%%%%%%
\usepackage{listings}
\newcommand{\xsource}[1]{\scriptsize{\textbf{Source:} {#1}}}
\newcommand{\source}[1]{\caption*{\xsource{#1}} }
\newcommand{\rsource}[1]{\vspace{-10pt} \caption*{\hfill \xsource{#1}} }
\newcommand{\tsource}[1]{\vspace{-10pt} \caption*{\xsource{#1}} }

% per le unità di misura
\usepackage[output-decimal-marker={,}]{siunitx}

\usepackage{xcolor}
\definecolor{darkgreen}{HTML}{007700}

% per il valore assoluto
\usepackage{mathtools}
\DeclarePairedDelimiter{\abs}{\lvert}{\rvert}

\lstset{
	basicstyle=\small\ttfamily,
	columns=fullflexible,
	keywordstyle=\color{violet}\bfseries,
	commentstyle=\color{darkgreen},
	breaklines=true,	 			% sets automatic line breaking
	captionpos=b,					% sets the caption-position to bottom
	stringstyle=\color{blue},     	% string literal style
	showstringspaces=false, 		% no special string spaces
	caption={\lstname},
	% title=\lstname,               % show the filename of files included with \lstinputlisting;
	numbers=left,
	numberstyle=\tiny,
	stepnumber=1,
	numbersep=5pt,
	frame=shadowbox
	% , float=*
}
%%%%%%%%%%%%%%%%%%%%%%%%%%%%

\usepackage{url}
\usepackage{xspace}
\usepackage{color}
\usepackage{booktabs}
\makeatletter
%%%%%%%%%%%%%%%%%%%%%%%%%%%%%% User specified LaTeX commands.
\usepackage{../manifest}

\makeatother

% https://en.wikibooks.org/wiki/LaTeX/Hyperlinks
% LaTeXimpaziente: "Il pacchetto hyperref, che di regola va caricato per ultimo, crea i collegamenti ipertestualivall’interno del documento, rendendo cliccabili i riferimenti incrociati"
\usepackage{hyperref}

%%%%%%%
 \newif\ifpdf
 \ifx\pdfoutput\undefined
 \pdffalse % we are not running PDFLaTeX
 \else
 \pdfoutput=1 % we are running PDFLaTeX
 \pdftrue
 \fi
%%%%%%%
 \ifpdf
 \usepackage[pdftex]{graphicx}
 \else
 \usepackage{graphicx}
 \fi
%%%%%%%%%%%%%%%
 \ifpdf
 \DeclareGraphicsExtensions{.pdf, .jpg, .tif}
 \else
 \DeclareGraphicsExtensions{.eps, .jpg}
 \fi
%%%%%%%%%%%%%%%

\newcommand{\java}{\textsf{Java}\xspace}
\newcommand{\android}{\texttt{Android}}
\newcommand{\dsl}{\texttt{DSL}}
\newcommand{\jazz}{\texttt{Jazz}}
\newcommand{\rtc}{\texttt{RTC}}
\newcommand{\ide}{\texttt{Contact-ide}}
\newcommand{\xtext}{\texttt{XText}}
\newcommand{\xpand}{\texttt{Xpand}}
\newcommand{\xtend}{\texttt{Xtend}}
\newcommand{\pojo}{\texttt{POJO}}
\newcommand{\junit}{\texttt{JUnit}}

\newcommand{\action}[1]{\texttt{#1}\xspace}
% \newcommand{\codescript}[1]{{\scriptsize{\texttt{#1}}}\xspace}
\newcommand{\codescript}[1]{{\mbox{\small{\texttt{#1}}}}\xspace}
\newcommand{\code}[1]{{\color{blue}\small{\texttt{#1}}}}
\newcommand{\fname}[1]{{\small{\color{magenta}\texttt{#1}}}}
\newcommand{\node}{\textsf{NodeJs}}
\newcommand{\qa}{\textsf{\textit{QActor}}\xspace}

% Cross-referencing
\newcommand{\labelsec}[1]{\label{sec:#1}}
\newcommand{\xs}[1]{\sectionname~\ref{sec:#1}}
\newcommand{\xsp}[1]{\sectionname~\ref{sec:#1} \onpagename~\pageref{sec:#1}}
\newcommand{\labelssec}[1]{\label{ssec:#1}}
\newcommand{\xss}[1]{\subsectionname~\ref{ssec:#1}}
\newcommand{\xssp}[1]{\subsectionname~\ref{ssec:#1} \onpagename~\pageref{ssec:#1}}
\newcommand{\labelsssec}[1]{\label{sssec:#1}}
\newcommand{\xsss}[1]{\subsectionname~\ref{sssec:#1}}
\newcommand{\xsssp}[1]{\subsectionname~\ref{sssec:#1} \onpagename~\pageref{sssec:#1}}
\newcommand{\labelfig}[1]{\label{fig:#1}}
\newcommand{\xf}[1]{\figurename~\ref{fig:#1}}
\newcommand{\xfp}[1]{\figurename~\ref{fig:#1} \onpagename~\pageref{fig:#1}}
\newcommand{\labeltab}[1]{\label{tab:#1}}
\newcommand{\xt}[1]{\tablename~\ref{tab:#1}}
\newcommand{\xtp}[1]{\tablename~\ref{tab:#1} \onpagename~\pageref{tab:#1}}
% Category Names
\newcommand{\sectionname}{Section}
\newcommand{\subsectionname}{Subsection}
\newcommand{\sectionsname}{Sections}
\newcommand{\subsectionsname}{Subsections}
\newcommand{\secname}{\sectionname}
\newcommand{\ssecname}{\subsectionname}
\newcommand{\secsname}{\sectionsname}
\newcommand{\ssecsname}{\subsectionsname}
\newcommand{\onpagename}{on page}

% What’s wrong with \bf, \it, etc.? --> https://texfaq.org/FAQ-2letterfontcmd
% \newcommand{\todo}[1]{\bf{TODO:}\emph{#1}}
\newcommand{\todo}[1]{\vspace{8px} \textbf{TODO:} \emph{#1}\\ \vspace{8px}}
\newcommand{\eng}[1]{\foreignlanguage{english}{\emph{#1}}}
\newcommand{\engl}[1]{\emph{#1}}
% \textit{ }

\newcommand{\xauth}{Filippo Frabetti}
\newcommand{\xemail}{filippo.frabetti@studio.unibo.it}
\newcommand{\xunibo}{Dipartimento di Informatica - Scienza e Ingegneria (DISI)\\Università di Bologna} % Alma Mater Studiorum
\newcommand{\xcourse}{Fondamenti di Intelligenza Artificiale M -- A.A. 2018/19}
\newcommand{\xtitle}{Attività progettuale su Genetic Programming}


% table of contents fino a subsubsections
\setcounter{tocdepth}{3}

% Mette i numeri di pagina nel piede, lasciando vuota la testatina
\pagestyle{plain}

% Tables: the height of each row is set to 1.5 relative to its default height
\renewcommand{\arraystretch}{1.5}
% The space between the text and the left/right border of its containing cell
\setlength{\tabcolsep}{6pt}

% %%%%%%%%%%%%%%%% BEGIN DOCUMENT %%%%%%%%%%%%%%%%
\begin{document}

\title{\xtitle}

\author{\xauth}

\institute{
  \xunibo
%  \email{\xemail}
}

% no titolo+autore nell'indice
{\def\addcontentsline#1#2#3{}\maketitle}
%\maketitle

\vspace{16px}
\begin{center}
\textbf{\xcourse}
\end{center}

% evita le righe eccessivamente lunghe aumentando la spaziatura tra le parole
% Il comando \fussy ripristina le impostazioni predefinite 
\sloppy

% indice sezioni
\tableofcontents
\newpage

%===========================================================================
\section{Introduzione}
\labelsec{overview}

Lo scopo di questa attività progettuale è quello di approfondire le tematiche relative all'utilizzo di tecniche di Genetic Programming per evolvere, in modo automatico, euristiche che possano guidare la ricerca nello spazio degli stati di un gioco a conoscenza completa, nella fattispecie il cosiddetto Gioco del Mulino o Schiera (\textit{Nine Men's Morris}).\\

Una prima parte dell'attività è dedicata all'investigazione dell'algoritmo di GP in generale, partendo dall'analisi di un'implementazione funzionante di questi (\emph{TinyGP}) per poi costruire una propria implementazione in versione Object-Oriented, in linguaggio \java.

Particolarmente rilevante a tal fine è la possibilità di effettuare semplici test di comparazione tra le due versioni dell'algoritmo, così da avere una forma di conferma empirica della correttezza dell'implementazione creata, conferma che sarebbe altrimenti difficile da ottenere data l'elevata componente stocastica presente nei processi di GP.\\

La seconda parte riguarda invece l'applicazione dell'algoritmo al Gioco del Mulino, con l'obiettivo di evolvere euristiche significative non basate su alcuna conoscenza strategica umana sul dominio del problema, e quindi prive di eventuali bias cognitivi. In particolare, speriamo di ottenere formule che diano, con buona approssimazione, un'indicazione sulla bontà di una mossa/di un dato stato della scacchiera a partire da una serie di dati in ingresso sullo stato stesso.

L'idea di fondo non è tanto quella di sviluppare giocatori particolarmente competitivi e intelligenti, quanto quella di dedurre quali siano le variabili che maggiormente influenzano lo stato della partita a seconda di dove queste compaiono all'interno delle euristiche migliori trovate. In linea di massima infatti, più una variabile si trova vicina alla radice, maggiore sarà il suo peso sul risultato finale e quindi sulla valutazione dello stato della partita. Viceversa, le variabili più ``periferiche'' avranno, relativamente, un'importanza minore.

\section{TinyGP}
\labelsec{tinygp}

\emph{TinyGP} è l'implementazione dell'algoritmo di Programmazione Genetica fornita nell'appendice B di \footnote{\url{http://www.gp-field-guide.org.uk/}} e originariamente sviluppata nell'ambito dell'omonima competizione tenutasi all'interno della Genetic and Evolutionary Computation Conference (GECCO) del 2004.

Per quanto sia soltanto uno dei tanti sistemi di Genetic Programming disponibili in rete, il suo pregio risiede nell'estrema compattezza e semplicità del codice e nella facilità di utilizzo, prestandosi quindi come un buon punto di partenza per capire le meccaniche di base dell'algoritmo e per effettuare alcuni semplici esperimenti.\\

Per essere precisi, l'implementazione pubblicata è la versione in \java di una precedente scritta in C per motivi di performance e di dimensioni dell'eseguibile risultante. Sebbene quindi formalmente si tratti di un programma \java, lo stile utilizzato è puramente procedurale e \emph{C-like}.

\subsection{Caratteristiche generali}
\labelssec{TinyGP_mainFeatures}

\begin{itemize}
\item Simboli terminali: un numero configurabile di variabili ($x_1$, \ldots, $x_n$) e di costanti casuali
\item Simboli di funzione: addizione, sottrazione, moltiplicazione e divisione protetta\footnote{Per evitare fallimenti a runtime, la divisione per zero viene intercettata prima della sua esecuzione (\emph{evaluation safety}).} (tutte funzioni binarie, a due argomenti)
\item Per la costituzione della popolazione iniziale viene utilizzato il metodo \emph{grow} con profondità massima configurabile
\item Il sistema è \emph{steady state}: il numero di individui per generazione è configurabile, ma costante; una generazione è considerata conclusa solo dopo che sono state effettuate tante operazioni genetiche quanti sono gli individui
\item La selezione avviene tramite \emph{tournament selection}, mentre tornei ``negativi'' vengono utilizzati per la scelta degli individui da sostituire
\item Operatori: \emph{subtree crossover} (con probabilità uniforme di selezionare un qualsiasi nodo) e \emph{point mutation}
\item Terminazione: ad ogni generazione, se la fitness del programma migliore è inferiore a una certa soglia ($10^{-5}$), l'algoritmo termina con successo, altrimenti termina con fallimento una volta raggiunto il numero massimo di generazioni impostato
\end{itemize}

\subsubsection{Fitness}
La fitness è data dall'opposto (``meno'') della somma delle differenze, in valore assoluto, tra l'output attuale e quello desiderato per ogni \emph{fitness case}.
I \emph{fitness case} vengono letti da file, uno per riga, e sono costituiti dall'assegnamento di un valore per ogni variabile e di un valore atteso per il risultato.

L'algoritmo di GP mira a massimizzare la funzione di fitness, ovvero a minimizzare l'errore totale tra i risultati attesi e quelli attuali.

\subsection{Parametri di configurazione e interfacce di utilizzo}
\labelssec{TinyGP_parameters}

Da linea di comando è possibile passare un \emph{seed} opzionale per il random number generator (se assente viene generato dal sistema) e il nome del file che contiene alcuni parametri dell'algoritmo e i \emph{fitness case}; se omesso viene ricercato un file chiamato ``problem.dat'' nel direttorio corrente.\\

\noindent La struttura del file di input è la seguente:
\begin{lstlisting}[caption={Struttura del file di input}]
NVAR NRAND MINRAND MAXRAND NFITCASES
V11 V12 ... V1n TARGET1
...
Vk1 Vk2 ... Vkn TARGETk
\end{lstlisting}
dove:
\begin{itemize}
\setlength\itemsep{0.1em}
\item[-] \texttt{NVAR} è il numero di variabili ($=n$, $x_1$ \ldots $x_n$)
\item[-] \texttt{NRAND} è il numero di costanti casuali da inserire come simboli terminali
\item[-] \texttt{MINRAND} e \texttt{MAXRAND} determinano il range da utilizzare per la generazione delle costanti casuali (\texttt{double})
\item[-] \texttt{NFITCASES} è il numero di \emph{fitness case} e di righe in aggiunta alla prima ($=k$)
\item[-] \texttt{Vij} è il valore da assegnare alla variabile $x_j$ ($j = 1, \ldots, n$) alla valutazione del \emph{fitness case} $i$ ($i = 1, \ldots, k$)
\item[-] \texttt{TARGETi} è il risultato atteso per il \emph{fitness case} $i$
\end{itemize}

\noindent Parametri di configurazione staticamente definiti all'interno del codice:
\begin{itemize}
\setlength\itemsep{0.1em}
\item \texttt{MAX{\_}LEN}: dimensione massima che un singolo programma può avere
\item \texttt{POPSIZE}: numero di individui in ogni generazione
\item \texttt{DEPTH}: profondità massima degli individui nella popolazione iniziale
\item \texttt{GENERATIONS}: numero massimo di generazioni
\item \texttt{CROSSOVER{\_}PROB}: probabilità di utilizzare l'operatore crossover, in alternativa alla mutazione
\item \texttt{PMUT{\_}PER{\_}NODE}: nel caso di mutazione, la probabilità che un singolo nodo venga mutato
\item \texttt{TSIZE}: numero di individui selezionati per ogni torneo
\end{itemize}

\noindent Ad ogni generazione vengono calcolate e stampate le seguenti statistiche:
\begin{itemize}
\setlength\itemsep{0.1em}
\item Numero della generazione corrente
\item Fitness e dimensione medie della popolazione
\item Individuo migliore e la sua fitness
\end{itemize}

\subsection{Dettagli dell'implementazione}
\subsubsection{Rappresentazione interna}
La rappresentazione interna utilizzata per gli individui/programmi non è altro che la linearizzazione della classica struttura ad albero tramite la notazione prefissa, dove ogni operatore è semplicemente seguito dai suoi operandi.
Ad esempio, la seguente espressione in notazione infissa:
\[
(3+4)*5
\]
può essere rappresentata senza ambiguità in notazione prefissa come:
\[
* + 3\:4\:5
\]
Mentre sotto forma di albero si avrebbe un nodo radice (*) che ha come figlio sinistro la somma di 3 e 4 (ovvero un sotto-albero $+$, con figli 3 e 4) e come figlio destro 5.\\

Grazie alla notazione prefissa è possibile fare a meno di parentesi e di regole di priorità degli operatori, ovvero non è nemmeno necessario mantenere, come sarebbe necessario fare nel caso di una rappresentazione ad albero, la distinzione dei diversi livelli di priorità. L'espressione $2+3*5$ infatti, pur non necessitando di parentesi, sotto forma di albero dovrebbe conservare in qualche modo la semantica di valutazione \textit{``prima moltiplicazioni e divisioni''}, generando un albero a due livelli ben diverso da quello prodotto da $(2+3)*5$.\\

La notazione prefissa ha quindi il vantaggio di poter mantenere in memoria pure sequenze di simboli, terminali o di funzione, riducendo ogni individuo ad un semplice array. Come vedremo, la scelta è di semplificare il più possibile le strutture dati di supporto, retaggio della versione in C, facendo uso di array di byte.

Naturalmente, appositi meccanismi vengono messi in atto per effettuare le operazioni di crossover e di mutazione sugli array, così come la generazione di individui (\emph{grow}) e la loro valutazione come espressioni.

\subsubsection{Simboli terminali e di funzione}
Come precedentemente accennato, la codifica utilizzata prevede di memorizzare ogni individuo/espressione come un array di byte (\texttt{char[]}). Per fare ciò occorre mappare tutti i simboli, terminali e di funzione, in un singolo byte ciascuno, per un totale di 256 possibili simboli.

L'idea è quella di utilizzare gli indici da 0 a 255 al posto dei \emph{valori} (nel caso delle costanti numeriche) e dei \emph{simboli} (nel caso delle variabili e degli operatori di funzione), così da effettuare tutte le operazioni genetiche sugli indici e ricorrendo ai valori riferiti quasi esclusivamente al momento della valutazione della fitness.\\

Il semplice individuo dell'esempio precedente, $* + 3\:4\:5$, in questa notazione basata su indici potrebbe quindi diventare qualcosa del tipo\footnote{043, 044 e 045 sono stati scritti con lo 0 davanti unicamente per evidenziare il fatto che non si tratta dei numeri 43, 44 e 45, ma di \emph{indici} che identificano dei valori numerici (in questo caso 3, 4 e 5). In \java la differenza è ancora più netta, in quanto questi ultimi sarebbero valori a virgola mobile a doppia precisione (\texttt{double}).}:
\[
112\;110\;043\;044\;045
\]
L'implementazione impiega, per scelta arbitraria, solamente i primi 114 indici sui 256 teoricamente disponibili. Il numero massimo di variabili e di costanti numeriche -- ovvero il numero massimo di simboli terminali -- è quindi vincolato a 110 ($=114-4$, dove 4 sono le funzioni presenti)\footnote{In effetti, per una probabile svista, il massimo numero ammesso risulta essere 109.}.\\

All'esecuzione del programma vengono generati in ogni caso, indipendentemente dal numero di variabili e di costanti impostato, 110 valori casuali nel range specificato da \texttt{MINRAND} e \texttt{MAXRAND}, per essere successivamente puntati dai relativi indici.

Da notare come di questi 110, solamente quelli con indice compreso tra \texttt{NVAR} e \texttt{NVAR+NRAND-1} saranno realmente utilizzati.

\subsubsection{Popolazione di partenza}
La prima generazione viene costruita tramite il metodo \emph{grow}, con una probabilità del 50\% di scegliere un terminale per ogni nuovo simbolo prima del raggiungimento della profondità massima impostata (\texttt{DEPTH}).

Nel raro caso che un individuo superi complessivamente la dimensione limite (\texttt{MAX{\_}LEN}) in termini di numero di nodi/simboli, questo viene scartato e ri-generato da capo.

\subsubsection{Valutazione della fitness}
Per quanto riguarda la fitness, per ogni \emph{fitness case} vengono copiati i valori -- letti da file -- da assegnare alle variabili nelle prime \texttt{NVAR} posizioni della tabella di conversione tra indici e costanti generate casualmente. In questo modo, all'esecuzione del programma il valutatore può ``dimenticarsi'' di quali indici rappresentino delle variabili e quali delle costanti, limitandosi a dover distinguere i soli indici che puntano a simboli di funzione.\\

Per l'esecuzione di una funzione vengono ricorsivamente prelevati dell'array, ed eventualmente calcolati, i suoi operandi.

Una menzione particolare merita di essere fatta per la divisione protetta ($/_p$), implementata nel seguente modo:
\[
num \: /_p \: den =
\begin{cases}
num & \text{se $\abs{den}\le0.001$} \\
num/den & \text{altrimenti}
\end{cases}
\]

La fitness viene immediatamente valutata per tutti i nuovi individui e memorizzata per impieghi futuri, in particolare per essere usata per effettuare le selezioni ``positive'' e ``negative'', per le statistiche relative ad ogni generazione e per la valutazione della condizione di terminazione.

\subsubsection{Ricombinazione e sostituzione}
Come menzionato nella sezione precedente, per mantenere le generazioni di dimensione costante, ogni nuovo individuo viene immediatamente inserito al posto di un altro già presente, selezionato tramite un torneo negativo. Non vi è dunque una netta separazione tra generazioni successive, nel senso che un individuo appena inserito all'interno della popolazione potrebbe essere scelto indistintamente come `genitore' di un altro individuo appartenente alla stessa generazione.

L'unica discriminante tra un individuo creato in una generazione piuttosto che in un'altra è quindi il solo conteggio delle operazioni genetiche effettuate, conteggio che si resetta ogni volta che si raggiunge il numero impostato come dimensione della popolazione.\\

La scelta, che premia la bassa occupazione di memoria e l'inserimento immediato di nuovi individui (potenzialmente migliori dei genitori), rende tuttavia il sistema più incline a soffrire di una possibile perdita di varietà dovuta proprio alle sostituzioni: una volta scelto un individuo tramite torneo negativo, infatti, questo viene perso per sempre anche se solo localmente peggiore di altri.

%\section{Old}
%Partendo dall'implementazione fornita in \footnote{\url{http://www.gp-field-guide.org.uk/}}, \emph{TinyGP}, un primo test consiste nell'eseguire il programma per approssimare, tramite Genetic Programming, la funzione seno in un certo intervallo (\texttt{[0, 6.2]}). Tuttavia, a differenza di quanto fatto in \footnote{\url{https://cswww.essex.ac.uk/staff/rpoli/gp-field-guide/B4CompilingandRunningTinyGP.html}}, si sceglie di esplicitare un \emph{seed} come primo argomento da riga di comando (nella fattispecie uguale a 1), così da consentire l'esatta replicazione dell'esperimento.
%
%Data la struttura del progetto, occorre collocarsi in \texttt{\{USER\_SPECIFIC\}/ProjectWork\_GP/NotSoTinyGP/bin} ed utilizzare il comando:
%\begin{lstlisting}[caption={Esecuzione di TinyGP da riga di comando}]
%java tinygp.tiny_gp 1 ..\resources\sin\sin-data.txt > ..\tinygp_seed1.out
%\end{lstlisting}
%Per praticità di consultazione, tutto ciò che viene stampato in output viene rediretto in un apposito file di testo.

\section{NotSoTinyGP}
L'implementazione \emph{NotSoTinyGP} mira a re-interpretare le logiche dell'algoritmo di Genetic Programming viste in \emph{TinyGP} in chiave Object-Oriented, spostando il focus non tanto sulla compattezza del codice e la sua efficienza a runtime, quanto sulla modularità e sull'estensibilità.

Si considerino ad esempio i casi in cui si volesse -- ed in parte è stato fatto -- aggiungere altri meccanismi per la generazione della popolazione iniziale, come i metodi \emph{full} e \emph{ramped half-and-half}, altri operatori genetici, altre funzioni oltre alle quattro operazioni elementari, eccetera.\\

Mentre da un lato si è cercato di mantenere per certi versi una forma di compatibilità con \emph{TinyGP}, specialmente per quanto riguarda l'aspetto delle interfacce di utilizzo, dall'altro si è scelto di approfittare della scrittura di questa nuova implementazione per sperimentare varianti relative ad alcune scelte progettuali, così che la messa a confronto delle due versioni non si limiti ad una semplice questione di efficienza e di variazioni stocastiche.

In ogni caso, le differenze tra le due implementazioni verranno riprese più nel dettaglio in seguito, in una sezione dedicata.

\subsection{Caratteristiche generali}
Si elencano solo quelle che si discostano da quanto dichiarato in \xss{TinyGP_mainFeatures} per \emph{TinyGP}.
\begin{itemize}
\item Simboli terminali e di funzione: ottenibili tramite un'apposita \texttt{NodeFactory}, pertanto facilmente configurabili
\item Per la costituzione della popolazione iniziale possono essere utilizzati il seguenti metodi: \emph{grow}, \emph{full}, \emph{ramped half-and-half} o versioni `ibride' di questi, ad esempio con una proporzione configurabile di \emph{grow} e \emph{full}
%\item Il sistema è \emph{steady state}: il numero di individui per generazione è configurabile, ma costante; una generazione è considerata conclusa solo dopo che sono state effettuate tante operazioni genetiche quanti sono gli individui
\item La selezione avviene tramite \emph{tournament selection}. Non vengono invece utilizzati tornei ``negativi'' per la scelta degli individui da sostituire
\item Operatori: \emph{subtree crossover} (con probabilità configurabile di selezionare un simbolo di funzione), \emph{point mutation}, \emph{subtree mutation} e riproduzione
%\item Terminazione: ad ogni generazione, se la fitness del programma migliore è inferiore a una certa soglia ($10^{-5}$), l'algoritmo termina con successo, altrimenti termina con fallimento una volta raggiunto il numero massimo di generazioni impostato
\end{itemize}

\subsection{Parametri di configurazione e interfacce di utilizzo}
Per quanto riguarda i parametri e le modalità di utilizzo, la scelta è di garantire, per quanto possibile, uniformità con quanto avviene in \emph{TinyGP}, mantenendo gli stessi argomenti da linea di comando e lo stesso formato per il file di configurazione contenente i \emph{fitness case}.

Viene considerato tuttavia inadeguato l'impiego di parametri definiti come campi statici, in quanto renderebbero necessaria una nuova compilazione ad ogni modifica nella configurazione, violando inoltre il principio di design secondo cui \emph{NotSoTinyGP} mira ad essere modulare e facilmente estensibile.\\

A questo proposito viene introdotto un ulteriore argomento opzionale da riga di comando: il nome di un file \texttt{Properties} consultabile a runtime e contenente tutti i parametri necessari.

\noindent Al momento è possibile invocare \emph{TinyGP} con i seguenti argomenti:
\begin{itemize}
\item file di configurazione
\item file di configurazione e seed
\end{itemize}
a questi viene aggiunto:
\begin{itemize}
\item file di configurazione, seed e file \texttt{Properties}
\end{itemize}

Inoltre, nei due casi precedenti, viene comunque ricercato nel direttorio corrente un file che termini con \texttt{.properties}: il primo trovato viene caricato nelle \texttt{Properties}, mentre eventuali altri saranno ignorati. Se anche questa ricerca dovesse risultare infruttuosa, allora ogni componente dell'algoritmo deve essere in grado di funzionare ugualmente ricorrendo a valori di default.

\subsection{Dettagli dell'implementazione}
\subsubsection{Rappresentazione interna}
La rappresentazione scelta per gli individui/programmi è quella tradizionale, ad albero, basata su una semplice gerarchia composta da \texttt{TerminalNode} (nodi terminali o foglie) e da \texttt{FunctionNode} (nodi intermedi o di funzione), entrambi sottoclassi di \texttt{Node}.

I riferimenti padre-figlio sono bidirezionali e la consistenza di tali collegamenti viene garantita dai relativi costruttori e metodi setter.

\subsubsection{Simboli terminali e di funzione}
Come precedentemente affermato, la generazione dei nodi viene affidata ad una factory, il cui compito è in sostanza quello di gestire gli insiemi dei simboli terminali e di funzione. %Le funzioni scelte sono, per semplicità, le stesse utilizzate da \emph{TinyGP}, ovvero le quattro operazioni elementari, e tra i terminali vi possono essere un numero configurabile di variabili.

Resta da definire è come garantire variabilità all'interno del \emph{terminal set} in termini di costanti numeriche. In questo caso, per differenziarsi dalla scelta fatta in \emph{TinyGP}, si è deciso di ricorrere ad una \emph{ephemeral random constant}, ovvero un simbolo-placeholder che, non appena selezionato, genera una nuova costante casuale (nell'intervallo presente nel file di input descritto in \xss{TinyGP_parameters}).\\

L'utilizzo di una \emph{ephemeral random constant} apre alcuni interrogativi su come gestire la cardinalità dell'insieme dei simboli terminali\footnote{Questa, insieme alla cardinalità dei simboli di funzione, può essere impiegata ad esempio nel metodo \emph{grow} per la generazione casuale di simboli prelevati dall'unione dei due insiemi.} e come la probabilità di scegliere, tra i terminali stessi, una variabile piuttosto che una costante, essendoci appunto \emph{un solo} simbolo corrispondente a costanti numeriche.

Entrambe le questioni sono state risolte considerando la compatibilità con le interfacce utilizzate da \emph{TinyGP}: poiché abbiamo a disposizione un parametro, \texttt{NRAND}, per indicare il numero di costanti casuali (e due per il loro range), questo viene utilizzato sia per stabilire la cardinalità dell'insieme, sommato al numero di variabili, sia per determinare la probabilità di scegliere una variabile ($\frac{nvar}{nvar+nrand}$). In questo secondo caso, se dovesse essere selezionata una costante numerica, questa viene generata casualmente ``al volo'' su richiesta; in altre parole, il sistema si comporta come se avesse a disposizione un insieme composto da \texttt{NRAND} \emph{ephemeral random constant}.

\subsubsection{Popolazione di partenza}
Per garantire una maggiore varietà nella popolazione iniziale, tra i diversi metodi disponibili, per i test viene utilizzato \emph{ramped half-and-half}, dove il 50\% degli individui vengono generati con \emph{full} e i restanti con \emph{grow}, in entrambi i casi sfruttando un range di profondità massime.\\

Una nota addizionale riguarda il metodo \emph{grow}: fissata una profondità limite, alla selezione di ogni nodo prima dell'ultimo livello (dove sono previsti solo nodi terminali) vi è una certa probabilità di scegliere una funzione piuttosto che un terminale. Come prima approssimazione si potrebbe pensare di determinare questa probabilità in base alla cardinalità relativa dei due insiemi, tuttavia ciò risulterebbe facilmente sbilanciato in molte situazioni reali in cui il numero di simboli terminali sovrasta quello delle funzioni.

Per lasciare aperte le porte a qualsiasi scenario di utilizzo, la scelta di come gestire la generazione di un nodo casuale viene delegata alle varie implementazioni di \texttt{NodeFactory}, senza che nulla di predefinito venga cablato all'interno dell'algoritmo \emph{grow}.

\subsection{Toy example: interi pari e dispari}
Una prima serie di test effettuati sull'implementazione creata, \emph{NotSoTinyGP}, sono stati svolti facendo ricorso ad un dominio estremamente semplice, costituito da un'unica funzione binaria (identificata dal simbolo ``\texttt{,}'', \textit{virgola}) e dagli interi tra 0 a 9 come simboli terminali.

In questo dominio, un individuo a profondità 2 potrebbe essere il seguente: \texttt{(3,(4,5))}, ovvero un nodo radice avente come figlio sinistro il numero 3 e come figlio destro il sotto-albero costituito dalla coppia 4 e 5.\\

L'obiettivo di questo ambiente giocattolo è unicamente quello di verificare il corretto funzionamento delle varie componenti fondamentali dell'algoritmo; a questo scopo sono stati predisposti dei semplici \texttt{main} di test relativi a: generazione della popolazione iniziale, crossover e mutazione\footnote{\texttt{TestInitialization}, \texttt{TestCrossover} e \texttt{TestMutation} nel package \texttt{test.mockimpl}}.

Un ulteriore programma di test ha invece permesso di testare l'intera procedura di GP nel suo complesso, con anche le fasi di selezione e valutazione della fitness, ottenuta dalla formula:

\[
f(N)=\alpha*\beta \quad
\text{con $\:\alpha=e^{-\abs{t-n_l}} \quad \beta=\frac{n_e}{n_l}$}
\]

Dove $t$ è il numero di simboli terminali desiderato (\texttt{target}), $n_l$ di quelli effettivi (\texttt{leaves}) e $n_e$ di quelli pari (\texttt{even}).

Come è facile osservare, la fitness di un individuo $N$ così calcolata è un numero reale compreso tra 0 e 1, massimo quando: $t=n_l=n_e$, ovvero quando ci sono esattamente \texttt{target} foglie, tutte contenenti numeri pari.\\

\noindent Il codice necessario per l'esecuzione dell'algoritmo viene riportato di seguito\footnote{Per il test è stata utilizzata un'istanza di \texttt{java.util.Random} con seed 1}:
\begin{lstlisting}[language=Java,caption={Estratto di \texttt{TestLauncher}}]
NodeFactory factory = new MockFactory(random, pTerm);
PopulationGenerator generator = new GrowGenerator(factory, depth);

FitnessFunction ff = new MockFitness(target);
SelectionMechanism sel = new TournamentSelection(random, ff, tsize);

Crossover stxo = new SubtreeCrossover(random);
Operator crossover = new BaseOperator(pCrossover,
	pop -> stxo.apply(sel.selectOne(pop), sel.selectOne(pop)));

Mutation ptm = new PointMutation(random, factory, pMutNode);
Operator mutation = new BaseOperator(pMutation,
	pop -> ptm.mutate(sel.selectOne(pop)));

Operator reproduction = new BaseOperator(pReproduction,
	pop -> sel.selectOne(pop).clone());

Node[] initialPop = generator.generate(popsize);
TimeMachine tm = new TimeMachine(random, new Operator[] { crossover, mutation, reproduction });

// termination: fitness of the best individual above a certain threshold
Predicate<Node[]> terminationCriterion = pop -> 
	ff.evalFitness(bestIndividual(pop,ff)) >= ff.maxFitness()-delta;
int gen = tm.run(initialPop, generations, terminationCriterion);

System.out.println("generation = " + gen + ", isSuccess = " + tm.isSuccess(terminationCriterion));
Node best = bestIndividual(tm.getCurrentGeneration(), ff);
System.out.println("best fitness = " + ff.evalFitness(best));
System.out.println("best node = " + best);
\end{lstlisting}

Tutti i parametri di configurazione presenti sono ricavati tramite un apposito file \texttt{Properties}, passato come argomento:
\begin{lstlisting}[caption={File mockimpl.properties}]
#same as TinyGP:
max_len=10000
popsize=100
depth=3
generations=100
crossover_prob=0.9
pmut_per_node=0.05
tsize=8

#others:
pTerm=0.5
target=8
mutation_prob=0.08
fitness_delta=0.05
\end{lstlisting}

\noindent Il risultato stampato a video è il seguente:
\begin{lstlisting}[caption={Output del programma}]
generation = 4, isSuccess = true
best fitness = 1.0
best node = (((2,((2,0),2)),((8,6),0)),0)
\end{lstlisting}
Dopo 4 generazioni è stato ottenuto un individuo dalla fitness massima, ovvero con esattamente 8 simboli terminali (\texttt{target=8}), tutti costituiti da numeri pari. Tale individuo può essere rappresentato in due dimensioni come:
\begin{verbatim}
N
| - 0
| - -
    | - -
    |   | - 0
    |   | - -
    |       | - 6
    |       | - 8
    | - -
        | - -
        |   | - 2
        |   | - -
        |       | - 0
        |       | - 2
        | - 2
\end{verbatim}

\section{Confronto tra implementazioni}
Al di là delle differenze puramente implementative, come la scelta della rappresentazione interna da utilizzare per gli individui, quello che risulta più significativo sono le differenze nell'algoritmo, ovvero le ragioni per le quali dovrei, sperabilmente, aspettarmi risultati diversi dall'esecuzione delle due versioni sul medesimo problema (a prescindere da ovvie variazioni stocastiche).\\

Dopo aver riassunto brevemente quelle che sono le differenze più significative dal punto di vista concettuale, le implementazioni verranno testate su due semplici problemi di regressione: tentare di approssimare una funzione polinomiale e una periodica (la funzione seno) in un dato intervallo; questo secondo esempio viene ripreso in quanto si tratta di quello proposto in \footnote{\url{https://cswww.essex.ac.uk/staff/rpoli/gp-field-guide/B4CompilingandRunningTinyGP.html}} per \emph{TinyGP}.

\subsection{Differenze concettuali}

\begin{table}
\begin{tabular}{l | p{3.7cm} | p{4.3cm}}
\textbf{Caratteristica} & \textbf{TinyGP} & \textbf{NotSoTinyGP} \\ \hline
Popolazione iniziale & \emph{grow} ($P_{term}=50\%$) & \emph{ramped half-and-half} (50/50 \emph{full}/\emph{grow}, $P_{term}=50\%$) \\ \hline
Costanti numeriche & set statico & \emph{ephemeral random constant} \\ \hline
Tornei negativi & sì, sostituzione immediata & no, generazioni distinte \\ \hline
Operatori & \emph{subtree crossover}, \emph{point mutation} & anche riproduzione (opt. \emph{subtree mutation}, \ldots)\\
\end{tabular}
\end{table}

%\begin{itemize}
%\item Generazione della popolazione iniziale: \emph{grow} con probabilità 50\% di selezionare simboli terminali; \emph{ramped half-and-half} con parametri configurabili (di fatto 50\%-50\% \emph{full}/\emph{grow} e 50\% di terminazione prima della profondità massima per \emph{grow})
%\item Costanti numeriche: definite staticamente ``una-tantum'' o generate su richiesta tramite \emph{ephemeral random constant}
%\item Sostituzione immediata degli individui nella popolazione; netta separazione tra generazioni
%\item Operatori: impiego di operatori diversi, a seconda di quelli disponibili nei due casi
%\end{itemize}

\subsubsection{Generazione della popolazione iniziale}
Gli individui della generazione 0 sono solitamente di scarso interesse ``pratico'', ovvero come candidate soluzioni al problema in oggetto: in primo luogo perché si tratta pur sempre di individui totalmente casuali, non derivanti da alcun processo di selezione, e in secondo luogo poiché generalmente di una dimensione troppo ridotta per poter sperare di approssimare con sufficiente precisione una ``buona'' soluzione.
Ciò che è rilevante è invece la \emph{varietà} della popolazione iniziale (e, di conseguenza, anche di quelle successive), poiché ciò garantisce di esplorare porzioni diverse dello spazio delle soluzioni, contrastando la tendenza a focalizzarsi su regioni localmente ottime tipica degli algoritmi di ricerca \emph{greedy}.

Sebbene da un lato il metodo \emph{ramped half-and-half} utilizzato in \emph{NotSoTinyGP} garantisca una maggiore varietà nella popolazione iniziale, è anche vero che questa stessa varietà può essere facilmente `recuperata' dopo poche generazioni a partire da una popolazione più omogenea, portando difficilmente a differenze significative tra i risultati ottenibili dalle due implementazioni.\\

Pur riconoscendone l'importanza ai fini del processo di GP, la varietà degli individui presenti nella popolazione iniziale potrebbe diventare veramente determinante in quei particolari casi in cui si è interessati a poche generazioni -- magari perché estremamente costose -- quindi si vuole evitare di pagare il prezzo di aspettare le ricombinazioni delle prime iterazioni per raggiungere un grado di diversificazione adeguato.

\subsubsection{Utilizzo di costanti numeriche}
La scelta di generare una volta per tutte le costanti numeriche che tutti i programmi andranno ad utilizzare non è solamente una questione di efficienza, ma consente anche di attribuire un chiaro significato alla cardinalità dell'insieme dei simboli terminali, concetto più difficile da delineare nel caso di \emph{ephemeral random constant}.
La conseguente minore varietà di costanti ad uso dei vari individui viene inoltre in parte appianata dalla possibilità di eseguire l'algoritmo di GP più e più volte, per poi tenere il risultato migliore.\\

Un aspetto interessante è come, avendo un numero limitato di simboli terminali a disposizione, risulti più facile effettuare semplificazioni e ricondursi a casi semplici anche con un alto numero di nodi per individuo (situazione tipica di generazioni avanzate).
Si pensi ad esempio alla funzione $2x+1$: questa potrebbe essere perfettamente rappresentata da un individuo a 2 livelli ($+\:*\:2\:x\:1$ in notazione prefissa), mentre per un individuo di profondità 7 la cosa non sarebbe affatto facile. Tuttavia, con le dovute semplificazioni di termini ``opposti'', al momento della valutazione si potrebbe arrivare ad ottenere un'espressione equivalente anche \emph{esattamente uguale} a $2x+1$.

\subsubsection{Selezione e sostituzione}
In entrambe le implementazioni il numero di individui per generazione è costante, tuttavia in \emph{NotSoTinyGP} non vengono effettuati tornei negativi per la sostituzione degli individui. Questo perché, per scelta, ogni generazione è rigidamente separata dalla precedente e dalla successiva, con la conseguenza che ogni individuo appartenente alla generazione $i$ può derivare unicamente da un'operazione genetica effettuata su uno o più individui della generazione $(i-1)$.\\

I tornei negativi aumentano la velocità di sostituzione degli individui peggiori, da un lato con l'ovvio effetto benefico di velocizzare la ricerca tra quelli invece più performanti, dall'altro riducendo potenzialmente la varietà presente all'interno della popolazione, limitando la ricerca nello spazio delle soluzioni.

Da notare come in ambedue le implementazioni vi sia una certa probabilità che un individuo ``sparisca'' senza lasciare alcuna progenie, se non dovesse vincere alcun torneo di selezione, mentre altri, quelli che dimostrano localmente di essere \emph{the fittest}, daranno vita a più nuovi individui della generazione successiva.

\subsubsection{Operatori}
Anche in questo caso il fatto di poter contare su un maggior numero di operatori genetici punta ad incrementare la varietà nella popolazione, e quindi a rendere più efficace il processo di ricerca. Questo però non significa che sia desiderabile introdurre un numero elevato di operatori diversi in modo arbitrario.

In linea teorica, dietro ad ogni operatore risiederebbero motivazioni almeno in parte \emph{domain dependent}: in altre parole, se si utilizza il \emph{crossover} è perché si fa la (forte) assunzione che, relativamente al problema in esame, la combinazione di due individui ad alta fitness dia, con una certa probabilità, un individuo la cui fitness è maggiore rispetto a quella che si avrebbe, probabilisticamente parlando, dalla combinazione di genitori con fitness più scarsa.
Poiché tuttavia i problemi a cui vengono applicate tecniche di GP sono solitamente così complessi da non consentire di avere basi teoriche sufficienti da giustificare a priori l'utilizzo di un operatore piuttosto che un altro, quello che spesso si fa è andare per tentativi: non conoscendo la migliore combinazione di operatori possibile, si cerca di utilizzarne vari, sperando che questi garantiscano una \emph{ragionevole} (ovvero non semplicemente casuale) ricerca nello spazio delle soluzioni.\\

Un operatore degno di menzione, impiegato in \emph{NotSoTinyGP} e presente in letteratura, è la riproduzione, ovvero la semplice copia di un individuo -- selezionato in base alla sua fitness -- da una generazione a quella successiva. Questo operatore, a priorità inferiore rispetto a \emph{crossover} e mutazione, punta a preservare, in un certo senso, buone combinazioni di geni invariate nel tempo, rinunciando a modifiche che potrebbero migliorarne ma anche peggiorarne la fitness.

\subsection{Approssimazione di una funzione polinomiale}
Un primo problema sul quale sono state testate le due implementazioni dell'algoritmo è quello di approssimare una funzione polinomiale, nella fattispecie la funzione $y(x)=2x^2+1$.

Per la valutazione della fitness è stato utilizzato il principio adottato da \emph{TinyGP}, ovvero l'opposto della somma degli errori, in valore assoluto, commessi su una serie di \emph{fitness case} forniti tramite uno specifico file di input\footnote{Descritto nella \xss{TinyGP_parameters} e in \url{https://cswww.essex.ac.uk/staff/rpoli/gp-field-guide/B2InputDataFilesforTinyGP.html}}, di cui viene riportata la prima riga:
\begin{lstlisting}[caption={Prima riga di \texttt{/NotSoTinyGP/resources/polynomial-data.txt}}]
1 2 0 5 21
\end{lstlisting}

\noindent Da interpretarsi nel seguente modo:
\begin{itemize}
\item Vengono impiegate una variabile e due costanti di valore compreso tra 0 e 5
\item Vi è un totale di 21 \emph{fitness case}, ovvero di coppie $(x ,y)$ corrispondenti ai valori interi della funzione per $x=0$ fino a $x=20$
\end{itemize}

Da notare come in \emph{NotSoTinyGP}, facendo uso di \emph{ephemeral random constant}, 2 rappresenti unicamente la probabilità ($2/3=66.66\%$) di selezionare una \emph{ERC} piuttosto che una variabile come simbolo terminale.

Il numero di costanti viene portato a 100 per \emph{TinyGP}, che invece genera preventivamente tutti i valori casuali di cui potrà disporre. La probabilità di selezionare variabili o costanti è data anche in questo caso dalla cardinalità dei due insiemi.\\

\noindent Gli altri parametri significativi vengono riassunti nella tabella \ref{tablePolynomialParams}.\\

\begin{table}
\begin{tabular}{l | p{2.7cm} | p{4.3cm}}
\textbf{Parametro}			& \textbf{TinyGP}	& \textbf{NotSoTinyGP}				\\ \hline
Max nr. generazioni			& 100 				& 100 								\\ \hline
Dim. popolazione			& 500 				& 500 								\\ \hline 
Profondità max iniziale		& 3 (\emph{grow}) 	& 1-3 (\emph{ramped half-and-half})	\\ \hline
Dim. tornei 				& 8 				& 8									\\ \hline
Prob. crossover 			& 0.9 				& 0.9 								\\ \hline
Prob. mutazione 			& 0.1 				& 0.08 (0.02 riproduzione)			\\ \hline
P. mutazione per nodo 		& 0.05 				& 0.05 								\\ \hline
Fitness delta 				& 0.05 				& 0.05 								\\
\end{tabular}
\caption{} \label{tablePolynomialParams}
\end{table}

In entrambi i casi è stato considerato un ``successo'' un individuo la cui fitness si trovi a meno di 0.05 (\texttt{fitness delta}) dal massimo teorico, ovvero 0, ``fallimento'' se dopo \texttt{max generazioni} la fitness migliore sia ancora al di sotto di questa soglia.

I grafici che seguono mostrano l'andamento, per ogni generazione, della dimensione %e della fitness
media degli individui e della fitness massima trovata.

%Vengono anche confrontate le distribuzioni della dimensione e della fitness degli individui delle popolazioni iniziali e finali, nei due casi.

\begin{figure}[!htb]
\centering
\includegraphics[width=0.7\textwidth]{../../NotSoTinyGP/runs/polynomial/avSize_1561974870333.png}
%\caption{ } \labelfig{ }
\end{figure}

\begin{figure}[!htb]
\centering
\includegraphics[width=0.7\textwidth]{../../NotSoTinyGP/runs/polynomial/maxFitness_1561974870333.png}
%\caption{ } \labelfig{ }
\end{figure}

Come risulta visibile dai grafici, già alla generazione numero 8 l'errore totale è sceso al di sotto dell'unità.
L'algoritmo termina infine con successo dopo sole 13 generazioni, ottenendo una funzione che si discosta dall'obiettivo (relativamente ai \emph{fitness case} considerati) di soli $0.0382$. 

\noindent La soluzione trovata, composta da 35 nodi, può essere semplificata in:
\[
2x^2 + 1.9 - \beta
\]
Dove $\beta$ è una polinomio razionale che tende a 1 ($\lim_{x \to \infty} \beta = 1$). Nel complesso, quindi, una buona approssimazione di $2x^2+1$.\\

Un'importante questione di cui dovremo tenere conto riguarda la dimensione degli individui nella popolazione, la cui media inizia a crescere vertiginosamente a partire dalla generazione 11. Una tale crescita incontrollata (\emph{bloat}) può facilmente portare il processo di GP ad essere computazionalmente ingestibile su un alto numero di iterazioni.\\

\noindent Vengono ora riportati i grafici relativi all'esecuzione con \emph{TinyGP}.

\begin{figure}[!htb]
\centering
\includegraphics[width=0.7\textwidth]{../../NotSoTinyGP/runs/tinygp/polynomial_pop500/avSize_1562518256383.png}
%\caption{ } \labelfig{ }
\end{figure}

\begin{figure}[!htb]
\centering
\includegraphics[width=0.7\textwidth]{../../NotSoTinyGP/runs/tinygp/polynomial_pop500/maxFitness_1562518256383.png}
%\caption{ } \labelfig{ }
\end{figure}

Come risulta evidente, la ricerca effettuata da \emph{TinyGP} è finita col concentrarsi unicamente su una zona sub-ottima, nella fattispecie sull'individuo $(x+x)*x = 2x^2$, tralasciando potenziali soluzioni migliori. Per contrastare questo problema è possibile aumentare la dimensione della popolazione: utilizzando campioni di 100000 individui (un incremento di 200 volte), infatti, il risultato diventa confrontabile con quello raggiunto da \emph{NotSoTinyGP}\footnote{Un numero così elevato di individui è computazionalmente tollerabile per l'estrema efficienza di \emph{TinyGP} e perché l'algoritmo si è fermato dopo poche generazioni, quindi con individui ancora di dimensioni molto contenute}.\\

\begin{figure}[!htb]
\centering
\includegraphics[width=0.7\textwidth]{../../NotSoTinyGP/runs/tinygp/polynomial_pop100000/avSize_1562518129305.png}
%\caption{ } \labelfig{ }
\end{figure}

\begin{figure}[!htb]
\centering
\includegraphics[width=0.7\textwidth]{../../NotSoTinyGP/runs/tinygp/polynomial_pop100000/maxFitness_1562518129305.png}
%\caption{ } \labelfig{ }
\end{figure}

L'algoritmo termina con successo dopo sole 5 generazioni, raggiungendo la fitness di $0.0003$ con un individuo di 17 nodi che, semplificato, risulta:
\[
2x^2+1.000014\ldots
\]

\subsection{Approssimazione della funzione seno}
Come già preannunciato, viene ripreso l'esperimento condotto nel testo \emph{A Field Guide to Genetic Programming} relativo all'approssimazione della funzione seno in un dato intervallo.\\

Le differenze relative ai parametri riportati nella tabella \ref{tablePolynomialParams} vengono riassunte nella tabella \ref{tableSinParams}.

\begin{table}
\begin{tabular}{l | p{2.7cm} | p{4.3cm}}
\textbf{Parametro}		& \textbf{TinyGP} 	& \textbf{NotSoTinyGP} 		\\ \hline
Prob. crossover 		& 0.9				& 0.7 						\\ \hline
Prob. mutazione			& 0.1 				& 0.28* (0.02 riproduzione) \\ \hline
P. mutazione per nodo 	& 0.05 				& 0.05 						\\ \hline
Min profondità pruning 	& N/A 				& 80 						\\ \hline
P. pruning per nodo 	& N/A 				& 0.5						\\ \hline
Fitness delta 			& 1.26 				& 1.26					 	\\ %\hline
\end{tabular}
\caption{} \label{tableSinParams}
\end{table}

Innanzitutto, dovendo considerare 63 \emph{fitness case}, in analogia al problema originario, si è deciso di alzare la soglia che determina quando la fitness di una soluzione sia accettabile come un successo a 1.26, ovvero tollerando un errore medio di 0.02 su ognuno dei 63 valori attesi per $y=sin(x)$.\\

Per rendere il problema trattabile dal punto di vista delle risorse di calcolo, è stato inoltre necessario introdurre un meccanismo per contrastare la crescita degli individui generazione dopo generazione. Per quanto esistano in letteratura diverse strategie possibili, la scelta è ricaduta sull'adozione di un nuovo operatore genetico ad-hoc, chiamato \emph{Prune Mutation}, in sostituzione alla normale mutazione.

L'idea alla base si fonda sull'assunzione che i nodi al di sotto di una certa profondità, contribuendo in minor peso al risultato, possano essere \emph{tagliati} per alleggerire gli individui molto grandi senza per questo perdere ``il grosso'' delle informazioni genetiche, conservate più vicino alla radice.

Al lato pratico, vengono introdotti due nuovi parametri: una soglia di profondità minima al di sopra della quale viene applicato il pruning e una probabilità, per ogni nodo sopra soglia, di essere tagliato (l'intero sotto-albero viene sostituito con un nuovo simbolo terminale).

Nel caso sia selezionata la mutazione, se un individuo non raggiunge la profondità minima necessaria, viene applicata la normale \emph{point mutation}, altrimenti scatta la \emph{Prune Mutation}.\\

\indent Vengono riportati i grafici relativi all'esecuzione di \emph{NotSoTinyGP} con in ingresso il file \texttt{/NotSoTinyGP/resources/sin-data2.txt}.
\begin{lstlisting}[caption={Prima riga di \texttt{/NotSoTinyGP/resources/sin-data2.txt}}]
1 2 -5 5 63
\end{lstlisting}

\begin{figure}[!htb]
\centering
\includegraphics[width=0.7\textwidth]{../../NotSoTinyGP/runs/sin/avSize_1562008988380.png}
%\caption{ } \labelfig{ }
\end{figure}

\begin{figure}[!htb]
\centering
\includegraphics[width=0.7\textwidth]{../../NotSoTinyGP/runs/sin/maxFitness_1562008988380.png}
%\caption{ } \labelfig{ }
\end{figure}

Il successo viene raggiunto alla 90-esima generazione, con una fitness di circa 1.24. Una seconda esecuzione senza alcun margine sulla fitness ha rivelato che dopo 100 generazioni l'errore complessivo è riducibile a 1.14, ovvero circa 0.018 in media sul singolo \emph{fitness case}.\\

L'effetto del pruning risulta evidente se si osserva l'andamento della profondità media e massima degli individui. In entrambi i grafici è infatti visibile un rallentamento in corrispondenza della soglia impostata (80).\\

\begin{figure}[!htb]
\centering
\includegraphics[width=0.7\textwidth]{../../NotSoTinyGP/runs/sin/avDepth_1562008988380.png}
%\caption{ } \labelfig{ }
\end{figure}

\begin{figure}[!htb]
\centering
\includegraphics[width=0.7\textwidth]{../../NotSoTinyGP/runs/sin/maxDepth_1562008988380.png}
%\caption{ } \labelfig{ }
\end{figure}

Vengono ora riportati per il confronto i grafici relativi all'esecuzione sul medesimo problema di \emph{TinyGP}.\\

Anche in questo caso, per evitare di incappare in un ottimo locale sono state considerate generazioni di 100000 individui, dal momento che test con popolazioni di dimensione 500 hanno portato la fitness massima a plateau ben lontani dalla soglia di 1.26.

Per quanto non vi sia, in \emph{TinyGP}, alcun meccanismo volto a limitare la dimensione delle soluzioni trovate, la terminazione con successo dopo poche generazioni ha di fatto evitato a monte il problema. Si tratta tuttavia di un caso eccezionale, sul quale non è possibile fare affidamento in generale.\\

\begin{figure}[!htb]
\centering
\includegraphics[width=0.7\textwidth]{../../NotSoTinyGP/runs/tinygp/sin_pop100000_fdelta1.26/avSize_1562520873555.png}
%\caption{ } \labelfig{ }
\end{figure}

\begin{figure}[!htb]
\centering
\includegraphics[width=0.7\textwidth]{../../NotSoTinyGP/runs/tinygp/sin_pop100000_fdelta1.26/maxFitness_1562520873555.png}
%\caption{ } \labelfig{ }
\end{figure}

Dopo 14 generazioni, la migliore soluzione trovata presenta una fitness attorno a 0.866, che corrisponde a 0.014 per \emph{fitness case}, valore confrontabile con la soluzione precedentemente ottenuta da \emph{NotSoTinyGP}.

Si noti tuttavia come la dimensione media per gli individui si attesti attorno ai 2000 nodi, ben al di sopra di quella riscontrabile in \emph{NotSoTinyGP} dopo 90 generazioni.

\subsection{Performance e tuning dei parametri}
Dall'esecuzione delle due implementazioni dell'algoritmo di GP sui problemi in esame è risultato evidente come vi siano dei vincoli piuttosto importanti in termini di performance per quanto riguarda la gestione di popolazioni molto numerose o con un elevato numero di nodi per individuo. In questo aspetto, con scarsa sorpresa, \emph{TinyGP} si dimostra in chiaro vantaggio per via delle scelte implementative legate alla rappresentazione interna degli individui.

Dal canto suo, \emph{NotSoTinyGP} può vantare una maggiore flessibilità derivante dalla possibilità di poter facilmente adottare diverse metodologie per l'inizializzazione della popolazione di partenza, nuovi operatori genetici, eccetera. In particolare, diventa fondamentale l'introduzione di meccanismi che limitino l'esplosione della dimensione delle candidate soluzioni al problema.\\

Il fatto di utilizzare popolazioni così grandi, in combinazione con il meccanismo di sostituzione tramite tornei negativi, senza un margine netto tra una generazione e l'altra, porta in parte a giustificare lo scarso numero di generazioni necessarie a \emph{TinyGP} per convergere verso ottime soluzioni. Venendo generati nuovi individui 200 volte più di frequente, per ogni generazione, rispetto a \emph{NotSoTinyGP}, il ``peso'' dietro ad ogni singola generazione risulta certamente alterato rispetto a quelle necessarie all'altra implementazione.\\

Al di là delle considerazioni appena enunciate, entrambe le implementazioni sono riuscite ad ottenere risultati tutto sommato comparabili, il che ci porta a concludere che -- una volta correttamente configurate -- la scelta tra le due possa essere guidata unicamente dalla natura del problema da risolvere: per problemi ``GP-standard'', per i quali è richiesta estrema velocità, footprint ridotto e una esplorazione più ampia possibile dello spazio delle soluzioni, \emph{TinyGP} risulta essere certamente il più indicato; viceversa, si può pensare di testare meccaniche più diversificate e complesse con il più strutturato, ma meno efficiente, \emph{NotSoTinyGP}.\\

Un altro aspetto da considerare è il processo, per niente banale, di tuning dei parametri. Oltre ai risultati descritti nelle sezioni precedenti, vari altri test più o meno soddisfacenti sono stati effettuati alla ricerca di una ``buona'' combinazione di parametri da utilizzare, insieme efficace e computazionalmente gestibile.

In questo come in molti altri algoritmi di ricerca, è chiaro come sia possibile ottenere risultati molto diversi con combinazioni diverse di parametri. Oltre alla dimensione della popolazione, è ad esempio possibile variare quella utilizzata per i tornei, aumentando o diminuendo la \emph{fitness pressure}, ovvero la probabilità che individui anche solo molto localmente migliori (o peggiori, per i tornei negativi) di altri vengano ripetutamente selezionati.

Infine, facendo largamente ricorso, per sua natura, a processi stocastici, qualsiasi algoritmo di GP è pensato non solo per essere eseguito più e più volte con gli stessi parametri (e diverso seed), ma anche con un ampio spettro di parametri diversi, così da poter selezionare il risultato globalmente migliore. Ciò può essere automatizzato ricorrendo ad algoritmi specializzati per il tuning dei parametri.

\section{Applicazione al Gioco del Mulino}
Il Gioco del Mulino (\emph{Nine Men's Morris}), o Schiera, è un gioco a conoscenza completa a due giocatori, il cui scopo è riuscire ad allineare tre pedine contigue sulla scacchiera, formando un Mulino, così da poter rimuovere una pedina avversaria. Il vincitore è colui che rende l'avversario impossibilitato a fare altri Mulini, lasciandolo con meno di 3 pedine in gioco o bloccando ogni sua possibile mossa lecita (sconfitta per ``soffocamento'').

Il gioco si alterna in tre fasi, con meccaniche leggermente diverse l'una dall'altra: nella prima ogni giocatore, a turno, posiziona una propria pedina in una qualsiasi posizione libera sulla scacchiera, nella seconda le pedine possono essere spostate in una casella adiacente, se libera, e nella terza e ultima fase in qualsiasi casella libera, contigua o meno.\\

L'elenco completo e dettagliato delle regole del gioco è facilmente reperibile on-line, al netto di alcune variazioni minori di carattere regionale\footnote{\url{https://it.wikipedia.org/wiki/Mulino_(gioco)}}.

\subsection{GP come iper-euristica}
Il problema relativo al Gioco del Mulino per il quale viene chiamata in causa la Programmazione Genetica è quello di trovare ``buone'' euristiche tra tutte quelle possibili, potenzialmente infinite, da impiegare per la scelta della (miglior) mossa da effettuare in una data situazione della scacchiera. Il punto è quindi riuscire a generare espressioni che approssimino molto bene il valore teorico che esprime lo stato della partita in termini di vantaggio/svantaggio relativo del giocatore nei confronti del suo avversario.

In questa sua applicazione, GP si classifica quindi come una iper-euristica, ovvero come meccanismo di ricerca nello spazio delle euristiche. Da \footnote{\url{https://cswww.essex.ac.uk/staff/rpoli/gp-field-guide/128GPtoCreateSearchersandSolversHyperheuristics.html}}:
\begin{quote}
The difference between meta-heuristics and hyper-heuristics is that the former operate directly on the problem search space with the goal of finding optimal or near-optimal solutions. The latter, instead, operate on the heuristics search space (which consists of the heuristics used to solve the target problem). The goal then is finding or generating high-quality heuristics for a problem, for a certain class of instances of a problem, or even for a particular instance.
\end{quote}
Sempre nella stessa pagina vengono presentati diversi casi in cui GP ha ottenuto ottimi risultati come iper-euristica.

\subsection{Valutazione della fitness: restrizione del problema}
Una volta che il processo di GP ha generato un individuo-euristica, occorre valutarne la fitness; per farlo, l'euristica trovata viene impiegata da un giocatore virtuale su una serie di partite: la proporzione di vittorie determina la bontà della soluzione.

Per limitare un possibile effetto di overfitting, ovvero il fatto che un individuo ottenga un punteggio alto solamente perché abile nello sconfiggere un determinato avversario -- o perché si è scontrato contro avversari più deboli della media --, il giocatore virtuale deve disputare un discreto numero (\texttt{nrMatches}) di partite contro avversari diversi, selezionati in modo casuale.\\

Perché ciò sia computazionalmente gestibile su popolazioni con un grande numero di individui e per molte generazioni, occorre limitare il Gioco del Mulino unicamente alla sua fase finale, ovvero quando uno dei due giocatori si ritrova ad avere in campo solo tre pedine ed è quindi libero di muoverle a piacimento sulle posizioni libere, senza vincoli di adiacenza.

Il numero delle pedine bianche e nere e la loro posizione sulla scacchiera al momento dell'inizio della terza fase di gioco viene determinato in modo casuale per ogni partita. Ovviamente, uno dei due giocatori avrà esattamente 3 pedine e l'altro un numero $>3$. Anche il primo a giocare viene sorteggiato caso per caso.\\

In aggiunta, per limitare la possibilità di ``cicli'', vengono imposti due vincoli addizionali che determinano la terminazione della partita:
\begin{itemize}
\item Un giocatore ripete la stessa mossa per \texttt{maxRepetitions} volte
\item Nessuna pedina viene rimossa dal gioco dopo \texttt{stalemate} mosse 
\end{itemize}
In entrambi i casi, il match si considera concluso in parità\footnote{Si tratta naturalmente di una semplificazione, altri approcci potrebbero valutare caso per caso il numero di pedine rimaste ai due giocatori, eccetera.}.

\subsection{Co-evoluzione: bianco VS nero}
Poiché non vi è una perfetta simmetria tra un giocatore che gioca come bianco o come nero (il nero è leggermente avvantaggiato), una euristica che si dimostri valida per un ruolo potrebbe non risultare altrettanto efficace per l'altro, il che suggerisce l'idea di separare il problema originario in due: la ricerca di una buona euristica per il bianco e di una per il nero.

Data la natura del problema, risulta sensato approcciarlo con il metodo della co-evoluzione: verranno evolute due popolazioni distinte, una per il bianco e una per il nero, e ad ogni generazione gli individui di una saranno utilizzati per la valutazione della fitness degli individui dell'altra, secondo le modalità descritte nella sezione precedente. Come diretta conseguenza di ciò, l'esito di un incontro andrà ad influire sulla fitness di entrambi i giocatori.

\subsection{Alcune considerazioni pratiche}
In linea teorica, un giocatore virtuale utilizzerebbe l'algoritmo \emph{Minimax} fino ad uno stato terminale del grafo degli stati del gioco, corrispondente a vittoria/sconfitta/pareggio, per poi propagare all'indietro il risultato trovato così da poter scegliere la mossa migliore da fare.

Al lato pratico, questo è solitamente infattibile per un problema di esplosione combinatoria, pertanto si fa affidamento sull'approssimazione ottenuta tramite un'euristica. Per incrementare l'affidabilità dell'approssimazione, un certo numero di livelli del grafo vengono comunque esplorati con \emph{Minimax}, così da poter effettivamente ``vedere avanti'' di $n$ mosse.\\

Nel nostro caso specifico, tuttavia, per preservare l'idea di partenza che prevede la distinzione tra le euristiche per il bianco da quelle per il nero, la scelta è quella di terminare la generazione del grafo dopo il primo livello, ovvero considerando unicamente tutti gli stati raggiungibili con ogni possibile mossa lecita a disposizione del giocatore.\\

\subsubsection{Terminazione}
Da notare come, conseguentemente alle limitazioni del gioco descritte nella sezione precedente, nessuno dei due giocatori può vincere per soffocamento: quello con 3 pedine non può bloccare abbastanza caselle, mentre l'altro non può impedire all'avversario di ``saltare'' da una parte all'altra della scacchiera. L'unico modo per vincere è rimuovere, formando dei Mulini, tante pedine avversarie da farne scendere il numero a 2.

Inoltre, detto $Np_{max}$ il numero di pedine del giocatore che ne ha di più, vi è un limite superiore al numero di mosse necessarie per concludere la partita:
\[
maxMosse = (Np_{max}-2) * (\texttt{stalemate}-1)
\]
Nel caso peggiore, infatti, è proprio il giocatore con più pedine a dover scendere sotto la soglia delle 3, e per farlo il suo avversario deve necessariamente impiegare meno di \texttt{stalemate} mosse per ogni pedina rimossa (senza lui perderne nemmeno una, essendo già a 3!).

\subsubsection{Simboli terminali e altri parametri}
Al fine di valutare quali variabili siano più significative per rappresentare lo stato della partita, vengono fornite una serie di variabili nell'insieme dei simboli terminali:
\begin{itemize}
\item $Nmul$, $Nmul_a$: numero di Mulini propri e dell'avversario
\item $Npair$, $Npair_a$: numero di coppie di pedine allineate, dove la terza casella è libera
\item $Nfree$, $Nfree_a$: numero di caselle libere occupabili con una mossa dalle proprie pedine/dalle pedine avversarie
\item $S_{xy}$: una variabile per ogni posizione $(x,y)$ sulla scacchiera, con valore 1/-1/0 se occupata da una pedina propria, avversaria o se libera
\end{itemize}

Si noti come sarebbe possibile introdurre un numero arbitrario di altre variabili, più o meno elaborate. Inoltre, come risulta evidente, vi sono dei legami tra le variabili menzionate, che non sono totalmente indipendenti. Altre informazioni sono invece derivabili dalla combinazione di più variabili, ad esempio il numero di pedine per ogni giocatore o quello delle pedine isolate.\\

TODO: altri parametri % se diventa troppo lunga, separa le sezioni

\subsection{Analisi e valutazione delle euristiche trovate}
%\subsection{Semplificazione e posizione relativa delle variabili}
% a che profondità (minima) si trovano le variabili nelle soluzioni migliori trovate?
% più sono vicine alla radice, più probabilmente influiscono sul risultato -> hanno peso/importanza maggiore

\section{Conclusioni}

\section{Sviluppi futuri}

%===========================================================================
% bibliografia
\clearpage

% vedi: Scrivere la tesi di laurea in LATEX (IntroTesi) pag. 30 (comandi per compilazione)
\addcontentsline{toc}{chapter}{\bibname}
% complessiva (tutto):
%\printbibliography

% vedi: L’arte di scrivere con latex (ArteLaTeX) pag. 136 in avanti
% i titoli e i filtri di queste sezioni sono definiti nel preambolo
\printbibliography[heading=principali,filter=papers]
\printbibliography[heading=web,type=online]
\printbibliography[heading=integrative,type=report]

\end{document}
